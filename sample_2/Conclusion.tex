\chapter{Заключение} \label{Concl}
В данной работе рассмотрена общая теория описания микроскопически неоднородных сверхпроводников вблизи перехода <<сверхпроводник-изолятор>>, и на основе представленного формализма приведён подход к описанию низкоэнергетических коллективных возбуждений в рассматриваемом классе систем.

Мотивацией исследования выступили многочисленные экспериментальные данные оптической спектроскопии, свидетельствующие о наличии в грязных сверхпроводниках коллективных возбуждений с энергией меньше ширины сверхпроводящей щели.

В \autoref{Theor} работы приводится подробный обзор уже имеющихся феноменологических, экспериментальных и теоретических сведений, существенных для построения общей модели сверхпроводников вблизи перехода <<сверхпроводник-изолятор>>. Далее на основе развитого формализма строится желаемая модель низкоэнергетических возбуждений и обсуждаются аспекты её применимости. 
\begin{itemize}
	\item Роль центрального объекта в этом описании играет псевдоспиновый гамильтониан \eqref{eq:Ham_final}, описывающий многочастичную квантовую механику преформированных электронных пар. В соответствующей части приводится полная феноменологическая база, приводящая к данной модели. Ключевыми феноменами, позволяющими рассматривать задачу таки образом, являются комбинация эффектов локализации одночастичных состояний за счёт сильного беспорядка и феномен преформирования электронных пар, сводящий динамику системы в релевантном диапазоне энергий к туннелированию электронных пар. Формулировка модели также требует принять ряд чисто качественных упрощений, призванных максимально упростить вызванную эффектами локализации сложную структуру одночастичных состояний и их перекрытий.
	\item С помощью семионного представления Попова-Федотова и методов функционального интегрирования приводится вывод основных уравнений сверхпроводимости --- уравнения самосогласования \eqref{eq:Order_paramter_self_consistency} и оператора, определяющего квадратичную часть функционала Гинзбурга-Ландау \eqref{eq:Order_parameter_fluctuation_inverse_propagator_explicit}. Существенным плюсом используемого подхода является имеющаяся возможность последовательно выяснять границы применимости стандартного среднеполевого формализма на основе нескольких вариантов диаграммной техники.
	\item В терминах квадратичной части функционала Гинзбурга-Ландау сформулирована задача, описывающая низкоэнергетические возбуждения в рассматриваемой системе.
	\item Сделано ключевое предположение о нерелевантности отклонения распределения параметра порядка от своего среднеполевого предела, которое позволяет свести задачу к изучению устройства спектра локального оператора \eqref{eq:Local_operator_definition}. Существенным является то, что использованное приближение лишает систему возбуждения Голдстоуна, являющегося следствием нарушения $U(1)$-симметрии.
\end{itemize}

В \autoref{Numer} описывается численный метод популяционной динамики, пригодный для изучения широкого класса задач локализации. 
\begin{itemize}
	\item Описаны теоретические аспекты устройства больших $K$-регулярных графов, связывающие их локальные свойства со свойствами этих величин на решётке Бете.
	\item В терминах производящего функционала представлен вывод метода популяционной динамики. Установлено, что данный метод является эквивалентном замкнутого интегрального уравнения на распределение диагональных элементов функции Грина линейного оператора на графе.
	\item Обсуждён правильный предельный переход для исследования спектральных характеристик именно задачи на $K$-регулярном графе. Порядок пределов определяется отсутствием у регулярного графа границы и вытекающими из этого условиями самосогласования.
	\item Продемонстрировано, что исследуемая задача благодаря конкретной структуре беспорядка в рамках используемых приближений эквивалентна хорошо изученной задаче локализации Андерсона.
	\item В работе также подробно обсуждены особенности поведения алгоритма популяционной динамики применительно к конкретной задаче. В частности, отмечено влияние внутренних параметров алгоритма, определяющее границы численной применимости всего метода.
\end{itemize}

Наконец, в \autoref{Result} приведена сводка полученных результатов численного исследования задачи, приведено подробное обсуждение этих результатов и намечен план дальнейших исследований.
\begin{itemize}
	\item В рамках проведённого исследования модель демонстрирует наличие делокализованных возбуждений во всём диапазоне параметров. Существенно, что этот факт расходится с утверждениями в работе \cite{FI_microwave}, предсказывавшими существования точки локализации состояний. Расхождения объясняются с привлечением задачи Андерсона, а именно, некорректно использованным приближением Anderson's Upper Limit. 
	\item Полученное распределение плотности состояний демонстрирует степенной хвост в широких пределах с характерных показателем степени $3/2$. Данное поведение приводит также к отличию средней и типичной плотности на три порядка. Результат находится в полном согласии с теории локализации Андерсона, так как при указанном выборе параметров модель всё время находится вблизи точки перехода c делокализованной стороны.
	\item Изучены зависимости характеристик распределения от числа ветвления графа $K$, на основании результатов построена фазовая диаграмма на Рис. \ref{fig:Phase_diagram}. Основными особенностями является видимое исчезновение состояний при очень малых $K \le 6$ и очень больших $K \gtrsim 160$ значениях параметра, а также явно присутствие точки смены асимптотик $K = 35$, степень физической значимости которой осталось невыясненной. Поведение модели на больших $K$ снова расходится с предсказаниями статьи \cite{FI_microwave}. Причиной расхождений является уже упомянутое приближение Anderson's Upper Limit, привёдшее к неверному предположению об устройстве состояний перед исчезновением.
	\item Проведена математически строгая оценка точного положения края спектра для данной задачи. Однако, численно значение оказывается на два порядка полученного численными методами. Расхождение объясняется конечной разрешающей способностью и метода и, что более важно, очень резким неаналитическим убыванием плотности состояний при приближении к краю спектра.
	\item В терминах исходной задачи низкоэнергетических возбуждений полученные результаты означают несостоятельность изучаемой модели, а значит, сделанных при её получении допущений. Ключевым предположением, приводящим к несостоятельным результатам, признано пренебрежение распределением и корреляциями параметра порядка на различных узлах.
	\item На основе полученных данных построены дальнейшие планы исследований. Основным направлением деятельности выделяется более тщательное изучения устройства распределения параметра порядка и связанные с этим модификации модели.
\end{itemize}

Автор данной работы выражает искреннюю благодарность своим научным руководителям Михаилу Викторовичу Фейгельману за многочисленные и в высшей степени информативные обсуждения данной задачи и Константину Сергеевичу Тихонову за неоценимую помощь в освоении всех технических аспектов теоретической и численной части данного исследования.
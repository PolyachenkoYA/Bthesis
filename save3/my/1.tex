\chapter{Описание эксперимнета по измерению объемного модуля упругости $K$}
\label{Sec:Exp}

В этой главе описывается эксперимент по измерению объемного модуля упругости кристаллического лизоцима. В дальнейшем эти экспериментальные точки будут использованы в качестве референсных значений для проверки корректности создаваемой МД модели. 

В разделе \ref{Subsec:AdsorbSurf} кратко излагаются результаты теории Brunauer-Emmett-Teller (BET) \cite{Boer1968}, описывающей процесс поверхностной адсорбции.

В разделе \ref{Subsec:AdsorbVol} показано применение теории Flory-Huggins \cite{Kuntz1974, Rowen1949}, которая описывает смешивание воды и молекул белка на 3-мерной сетке.

В разделе \ref{Subsec:AdsorbExp} описаны детали ЯМР-эксперимента, результаты которого далее используются для измерения модуля упрогости белкового кристалла.

В разделе \ref{Subsec:K} представлено описание экспериментальных данных моделью Flory-Huggins, позволяющее извлечь значения модуля упругости.

Гидратация играет важную роль в свертывании белков, их динамике и функциях [1-4]. Например ферментативная активность лизоцима заметно возрастает при уровнях гидратации выше $h = 0.2$ (в граммах воды на грамм сухого белка) [2, 5]. Взаимодействие белков с водой также было одной из центральных тем при изучении свертваниях белка с того момента как концепция гидрофобного взаимодейтвия была введена Каузманов в 1959 [1].

Для лизоцима уровень $h = 0.2$ соответствует давлению пара $P/P_0 \sim 0.7$, где $P_0$ -- давление насыщенного пара при данной температуре. При дальнейшем насыщении происходит дополнительное поглощение воды в сравнении с линейным ростом при $P/P_0 < 0.7$. Опубликованы работы, связывающие именно это изменение в поведении лизоцима с его ключевыми функциями [2,13]. На данный момент не существует консенсуса относительно механизма этого дополнительного поглощения [2,6]. 

Гравиметрический метод, широко применяемый для исследования белков в физиологическом диапазоне температур, может быть неудобен для работы при температурах ниже комнатной и приближающихся к 0 $C^{\circ}$. Поэтому в данной экспериментальной работе уровень гидратации измерялся с помощью ядерного магнитного резонанса водородов 1Н в системе [15,16].

\section{Поверхностная адсорбция}
\label{Subsec:AdsorbSurf}



\cite{Gambosh:1951} 

\section{Модель Flory-Huggins}
\label{Subsec:AdsorbVol}


\section{Эксперимент}
\label{Subsec:AdsorbExp}


\section{Описание эксперимента моделью Flory-Huggins}
\label{Subsec:K}


\chapter{Построение МД модели кристалла Лизоцима}
\label{Sec:Calculation}

В предыдущей главе были приведены все необходимые теоретические сведения для дальнейшего численного решения краевых задач модели Томаса-Ферми с поправками. В этой главе будет показано, как реализован алгоритм расчёта.

В предыдущей главе был описан метод, которым были получены референсные значения укругости $K$, согласовавшиеся с ранее опубликованными значениями.

В разделе~\ref{Subsec:CalculationPrep} описана подготовка кристалла лизоцима из базы PDB к симуляции. Далее в \ref{Subsec:CalculationConverge} продемонстрирована сходимость количества добавляемой в белок воды по некоторым базовым параметрам. Наконец, в \ref{Subsec:Calculation2ways} описаны 2 используемых метода расчета модуля упругости по имеющимся МД траекториям.

\section{Подготовка PDB-белка к расчету}
\label{Subsec:CalculationPrep}

1

\section{Сходимость уровня гидротации}
\label{Subsec:CalculationConverge}

2

\section{2 метода расчета модуля упругости $K$ по траекториям}
\label{Subsec:Calculation2ways}

3

\intro

\textbf{Актуальность работы.} Всестороннее изучение белков может быть полезно в широком круге задач, стоящих сейчас перед человечеством, т.к. белки - тип биомолекул, выполняющих основную часть функций в клетке. Лучшее понимание устройства белков может способствовать, например, разработке новых методов исправления их неправильной работы, что значило бы излечение многих серьезных болезней. На сегодняшний день существуют методы определения последовательности белков, работающие по большей части в автоматическом режиме и не требующие больших затрат ресурсов. Но для глубокого понимания устройства белка необходимо знать не только его аминокислотную последовательность, но и трехмерную структуру. Метод кристаллографии, считающийся сейчас классическим для определения 3D-струкруты белков, уже менее тривиален в реализации. Одна из главных сложностей в нем - необходимость кристаллизации белка для его исследования. Кристаллизация белков часто сопряжена с созданием необычных и при этом строго контролируемых физических условий, что уже говорит о сложности процесса. Иногда же в силу особенностей конкретной молекулы ее кристаллизация вообще не представляется возможной. В таких случаях кристаллизация может быть проведена с заменой частей белка, мешающих ей. Это опять же многократно усложняет процесс, т.к. нужно предпринимать попытки кристаллизации многих подобных белков. Методы молекулярного моделирования активно используются для поиска мутаций, способствующих кристаллизации белка. Однако возможная роль молекулярного моделирования не исчерпывается этими вопросами.

\textbf{Цель работы} состояла в том, чтобы довести молекулярно-динамическую модель кристаллического Лизоцима до приемлимого воспроизведения экспериментальных измерений его модуля упругости. Имея такую модель, можно будет интерпретировать результаты экспериментов с молекулярно-атомарной точки зрения. Также такая модель заметно упростит исследование лизоцима в различных условиях, т.к. не будет необходимости каждый раз проводить экспериментальные измерения. Наконец, подход создания МД модели, разработанный для кристаллического лизоцима, может быть модифицирован для создания МД моделей более сложных белков, таких как Гемоглобин. Это может помочь найти для таких белков мутации, способствующие кристализации, т.к. опять же не будет необходимости экспериментально проверять множество различных малых мутаций белка в различных физических условиях.

Работа состоит из трёх глав и заключения.

В главе \ref{Sec:Exp} описывается эксперимент по измерению объемного модуля упругости $K$ кристаллического лизоцима, результаты которого в дальшейшем используются как референсные значения.

В главе \ref{Sec:Calculation} рассказывается об этапах построения МД модели

В главе \ref{Sec:Results} приведены основные результаты численного моделирования.

В заключении кратко обсуждаются полученные результаты и делаются соответствующие выводы. Изложены также перспективы развития работы.
